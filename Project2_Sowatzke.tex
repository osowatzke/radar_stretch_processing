\documentclass[conference]{IEEEtran}
\IEEEoverridecommandlockouts
\usepackage{matlab-prettifier}
\usepackage{cite}
\usepackage{amsmath,amssymb,amsfonts}
\usepackage{nccmath}
\usepackage{algorithmic}
\usepackage{graphicx}
\usepackage{textcomp}
\usepackage{xcolor}
\usepackage{float}
\usepackage{tabularx}
\def\BibTeX{{\rm B\kern-.05em{\sc i\kern-.025em b}\kern-.08em
    T\kern-.1667em\lower.7ex\hbox{E}\kern-.125emX}}

\begin{document}

\title{Project 2 : Range Doppler Matrix Generation for an LFM Radar System}

\author{\IEEEauthorblockN{Owen Sowatzke}
\IEEEauthorblockA{\textit{Electrical Engineering Department} \\
\textit{University of Arizona}\\
Tucson, USA \\
osowatzke@arizona.edu}}
\maketitle

\begin{abstract}
The range resolution of a radar system is inversely proportional to the bandwidth of its transmitted waveform. To achieve a fine range resolution, a high bandwidth waveform is required. This typically results in an increased ADC sampling rate, which leads to higher production costs. A linear frequency modulated (LFM) waveform can be used in conjunction with stretch processing to provide a large transmitted waveform bandwidth with a reduced ADC sampling rate. This leads to reduced production costs and increased consumer availability.   
%The range resolution of a rectangular pulse is inversely proportional to the length of the rectangular pulse. Therefore, to achieve fine range resolution, a short rectangular pulse is required. However, decreasing the pulse length also leads to a reduction in SNR. Pulse compression waveform seek to decouple pulse length from range resolution and in turn make it possible to simultaneously achieve fine range resolution and a high SNR. One common pulse compression waveform is a linear frequency modulated (LFM) waveform. This document simulates the matched filer response of an LFM waveform and examines the resulting SNR and range resolution.
\end{abstract}

\begin{IEEEkeywords}
Pulse Compression, Linear Frequency Modulated (LFM) Waveform, Matched Filter, SNR, Range Resolution
\end{IEEEkeywords}
\section{Introduction}
%A linear frequency modulated (LFM) waveform is a pulse compression waveform defined by the following equation:
%\begin{equation}
%x(t)=e^{j\pi\beta t^2/\tau}
%\label{lfm_waveform}
%\end{equation}
%where:
%\begin{fleqn}[\parindent]
%\begin{align*}
%\beta &= \text{Waveform Bandwidth (Hz)}\\
%\tau &= \text{Waveform Duration (s)}
%\end{align*}
%\end{fleqn}
%A sample pulsed radar system will be created in MATLAB to simulate the matched filter response of this waveform. Using the radar system model, the SNR of the matched filter output will be measured and compared to the SNR of the input signal. Additionally, multiple target returns will be generated and used to measure the system's range resolution.
%\section{Sample Radar System Parameters}
%To generate the required matched filter response, a pulsed radar system was simulated in MATLAB using the following set of parameters: 
%%The pulse radar system simulated in MATLAB was designed with the following set of parameters: 
%\begin{table}[H]
%\caption{Radar System Parameters}
%\label{Parameter Table}
%\begin{tabularx}{0.5\textwidth}{| X | X |}
%\hline
%Carrier Frequency & 10 GHz \\
%\hline
%Sample Rate & 100 MHz \\
%\hline
%Transmit Power & 20 dB \\
%\hline
%Antenna Gain & 44.15 dB \\
%\hline
%Noise Figure & 10 dB \\
%\hline 
%System Losses & 5 dB \\
%\hline
%PRF & 50 kHz \\
%\hline
%Duty Cycle & 20\% \\
%\hline
%Chirp Bandwidth & 100 MHz \\
%\hline
%\end{tabularx}
%\end{table}
%\noindent
%Additional parameters can be computed using the parameters given in Table \ref{Parameter Table}. For example, the radar system's PRI can be computed as follows:
%\begin{equation}
%PRI = \frac{1}{PRF} = 20 {\mu}s
%\label{PRI Equation}
%\end{equation}
%Another parameter of interest, the length of the LFM pulse ($\tau$), can be computed using the duty cycle ($D$) as follows:
%\begin{equation}
%\tau = PRI \cdot D = 4 {\mu}s
%\label{tau equation}
%\end{equation}
%\section{Modeling the Matched Filter Output}
%\label{modeling_section}
%To model the matched filter response of the system, the received signal must first be generated for each of the target returns. The process of generating the received signal is illustrated in Fig. \ref{gen_rx_sig}.
%\begin{figure}[H]
%\centerline{\fbox{\includegraphics[width=0.3\textwidth]{gen_rx_sig.png}}}
%\caption{Generating the Received Signal.}
%\label{gen_rx_sig}
%\end{figure}
%Each of the target returns is a delayed and scaled version of the transmitted signal. As such, the transmitted LFM waveform must be generated as part of the workflow. This is done using equation \eqref{lfm_waveform}. Note that the transmitted signal must be padded with zeros to account for the listening period after transmission.
%\par
%For a target at a range of $R$ meters, the transmitted signal is delayed by
%\begin{equation}
%t_d = \frac{R}{2c} \enspace s
%\end{equation}
%The SNR of the target return is given by the following equation:
%\begin{equation}
%SNR = \frac{P_t G^2 \lambda^2 \sigma}{(4\pi)^3 R^4 k T_0 B_n F_n L_s L_\alpha(R)}
%\label{snr_formula}
%\end{equation}
%To create a target return with the desired SNR, either the signal power or noise power can be scaled. For this simulation, the signal power is scaled while the noise power is held constant. The transmitted signal given in equation \eqref{lfm_waveform} has unit amplitude, and the noise is sampled from the standard complex normal distribution. Therefore, to achieve the desired SNR, the target return must be scaled by the following factor:
%\begin{equation}
%k = 10^{SNR(dB)/20}
%\end{equation}
%\par
%The received signal is the sum of each target return and noise. During waveform transmission, the receiver is assumed to be "blanked" (i.e. it cannot listen while transmitting). To model the effects of "blanking", the beginning of the received pulse is replaced with zeros.
%\par
%To increase the SNR of the received signal, the signal is passed through a matched filter. The radar system's matched filtering process is shown in Fig. \ref{gen_mf_output}. 
%\begin{figure}[H]
%\centerline{\fbox{\includegraphics[width=0.15\textwidth]{gen_mf_output.png}}}
%\caption{Matched Filtering of the Input Signal.}
%\label{gen_mf_output}
%\end{figure}
%\par
%The system's matched filter is generated using to the following equation:
%\begin{equation}
%h = x^*(T_M-t)
%\end{equation}
%To ensure the matched filter is causal, $T_M$ is chosen to be the length of the transmitted LFM waveform.
%\par
%In the radar system model, both the noise and target returns are filtered independently. Note that this provides the same result as directly filtering the received signal.
%\begin{equation}
%y(t) = h(t)*(x(t)+n(t)) = h(t)*x(t) + h(t)*n(t)
%\end{equation}
%However, it also provides the ability to easily extract signal and noise power measurements from the matched filter output.
%\section{Matched Filter Output SNR}
%\label{mf_snr_section}
%The SNR at the output of the matched filter output should be greater than the SNR of the received signal by a factor of $L$, where $L$ is the matched filter length in samples. For the given radar system, the matched filter length is
%\begin{equation}
%L = \frac{\tau}{T_s} = \tau f_s = 400 \text{ samples}
%\end{equation}
%Therefore, the SNR of the matched filter output should be $10log_{10}(400)=26.02 \text{ dB}$ greater than that of the received signal. This result will be confirmed via simulation.
%\par
%To simulate the matched filter response of the system, a target is placed in the first ambiguity at a range corresponding to a time delay of 25\% of the PRI. The resulting range is
%\begin{equation}
%R = \frac{ct_d}{2} = 750 \text{ m}
%\label{tgt1 range}
%\end{equation}
%The RCS of the target can be calculated to provide an arbitrary SNR at the ADC. Starting with equation \eqref{snr_formula}, the RCS can be expressed in the following form:
%\begin{equation}
%\sigma = \frac{(4\pi)^3 R^4 k T_0 B_n F_n L_s L_\alpha(R)\cdot SNR}{P_t G^2 \lambda^2}
%\label{rcs_formula}
%\end{equation}
%For this system, the SNR at the ADC is chosen to be 20 dB. This SNR corresponds to a RCS of $0.01307m^2$.
%\par
%The received signal for this target configuration can be generated according to Section \ref{modeling_section}. The resulting ADC signal is shown in Fig. \ref{adc_sig}.
%\begin{figure}[H]
%\centerline{\fbox{\includegraphics[width=0.4\textwidth]{adc_sig.png}}}
%\caption{Received ADC Signal.}
%\label{adc_sig}
%\end{figure}
%\noindent
%This ADC signal can then be passed through the system's matched filter to produce the plot shown in Fig. \ref{mf_output}.
%\begin{figure}[H]
%\centerline{\fbox{\includegraphics[width=0.4\textwidth]{mf_output.png}}}
%\caption{Matched Filter Output.}
%\label{mf_output}
%\end{figure}
%\noindent
%The SNR of the matched filter output is given by the following equation:
%\begin{equation}
%SNR = 10log_{10}\frac{P_s}{P_n}
%\end{equation}
%where:
%\begin{fleqn}[\parindent]
%\begin{align*}
%P_s &= \text{Signal Power}\\
%P_n &= \text{Noise Power}
%\end{align*}
%\end{fleqn}
%The resulting signal power is defined using the peak of the filtered target return as follows:
%\begin{equation}
%P_s = \text{max(}|x[k]|\text{)}^2
%\end{equation}
%The resulting noise power is defined using the filtered noise samples ($n[k]$). Note that the first $2L-1$ noise samples are ignored due to "blanking" and "charge-up" of the matched filter. Let $m \in [0, N)$ define the "charged-up" noise samples of the matched filter. Then, the noise power is given by
%\begin{equation}
%P_n = \frac{1}{N}\sum_{m=0}^{N-1}|n[m]|^2
%\end{equation}
%\par
%For the given simulation, the observed SNR at the matched filter output is $45.97 \text{ dB}$. Note that this is approximately equivalent to the expected SNR of $46.02 \text{ dB}$.
%\section{Matched Filter Range Resolution} 
%The range resolution of the matched filter output is theoretically given by
%\begin{equation}
%\Delta R = \frac{c}{2\beta} = 1.5 \text{ m}
%\end{equation}
%This result will be confirmed via simulation. First, a target will be placed as described in Section \ref{mf_snr_section}. Then, a second target will be placed at a larger range and moved toward the first target until the two targets are indistinguishable from each other. 
%\par
%To start, the second target was placed two code lengths ($2L$ samples) away from the first target. This guarantees that the matched filter responses from each target return are non-overlapping. The resulting range of the second target is given by
%\begin{equation}
%R = 750 + \frac{c(2\tau)}{2} = 750 + c\tau = 1950 \text{ m}
%\end{equation}
%The SNR of the second target return is configured to be 20 dB. Using equation \eqref{rcs_formula}, the resulting RCS is given by $0.59758m^2$. The matched filter response for the combined target return is shown in Fig. \ref{mf_output_clearly_resolvable}.
%\begin{figure}[H]
%\centerline{\fbox{\includegraphics[width=0.4\textwidth]{clearly_resolvable.png}}}
%\caption{Resolvable Target Returns ($1200$m Spacing).}
%\label{mf_output_clearly_resolvable}
%\end{figure}
%\par
%The second target is clearly distinguishable from the first. As such, the second target can be moved significantly closer to the first target. If it is placed 2 ADC samples away from the first target, the resulting range is given by
%\begin{equation}
%R = 750 + \frac{c(2T_s)}{2} = 750 + cT_s = 753 \text{ m}
%\end{equation}
%The SNR of this target return can be configured as 20 dB. This results in an RCS of $0.01328m^2$. The matched filter response resulting for the updated target returns is shown in Fig. \ref{resolvable_full}.
%\begin{figure}[H]
%\centerline{\fbox{\includegraphics[width=0.4\textwidth]{resolvable_full.png}}}
%\caption{Resolvable Target Returns ($3$m Spacing).}
%\label{resolvable_full}
%\end{figure}
%\noindent
%After zooming in on the peak of the matched filter response, the targets can easily be resolved. This is shown in Fig. \ref{resolvable_peak}.
%\begin{figure}[H]
%\centerline{\fbox{\includegraphics[width=0.4\textwidth]{resolvable_peak.png}}}
%\caption{Resolvable Target Returns ($3$m Spacing).}
%\label{resolvable_peak}
%\end{figure}
%\par
%If the second target is moved one sample closer to the first target, the range of the second target is given by 
%\begin{equation}
%R = 750 + \frac{cT_s}{2} = 751.5 \text{ m}
%\end{equation}
%If the SNR of the target is configured as 20 dB, the resulting RCS is given by $0.01317m^2$. The peak of the updated matched filter response is shown in Fig. \ref{unresolvable_peak}.
%\begin{figure}[H]
%\centerline{\fbox{\includegraphics[width=0.4\textwidth]{unresolvable_peak.png}}}
%\caption{Unresolvable Target Returns ($1.5$m Spacing).}
%\label{unresolvable_peak}
%\end{figure}
%\par
%Note that the target returns have blurred into a single target return. As such, the resulting range resolution is $1.5 \text{ m}$. The measured range resolution is identical to the theoretical range resoution. Therefore, the simulated result confirms the theoretical result.
%\section{Conclusion}
%This document examined the matched filter response of an LFM waveform using a simulated pulsed radar system. Quantities of interest included the SNR and range resolution of the matched filter output. The SNR of the matched filter output was greater than than input SNR by a factor of the code length ($L$). Additionally, the range resolution of the matched filter output was $c/(2\beta)$.
%%A radar system was simulated in MATLAB and used to generate the matched filter response of an LFM waveform. 
%%The measured SNR gain from the matched filter was the code length ($L$), and the measured range resolution was $c/(2\beta)$.
%\par
%For a rectangular pulse of the same length $\tau$, the increase in SNR due to the matched filter is also given by $L$. However, the range resolution is increased to $c\tau/2$. Therefore, the range resolution of an LFM waveform is smaller than that of a rectangular waveform by a factor of $\beta\tau$. For the simulated radar system, $\beta\tau = 400$. This result demonstrates the advantage of using a pulse compression waveform such as LFM. Specifically, the SNR of the system can be increased, while maintaining a fine range resolution.
%
%\onecolumn
%\pagebreak
%\appendices
%\section{MATLAB Source Code}
%\label{matlab_code}
%\lstset{style=Matlab-editor}
%\lstinputlisting{Project1_Sowatzke.m}
%\raggedbottom
%\pagebreak
%\lstinputlisting{radar.m}
%\raggedbottom
\end{document}
