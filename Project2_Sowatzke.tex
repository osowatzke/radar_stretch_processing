\documentclass[conference]{IEEEtran}
\IEEEoverridecommandlockouts
\usepackage{matlab-prettifier}
\usepackage{cite}
\usepackage{amsmath,amssymb,amsfonts}
\usepackage{nccmath}
\usepackage{algorithmic}
\usepackage{graphicx}
\usepackage{textcomp}
\usepackage{xcolor}
\usepackage{float}
\usepackage{tabularx}
\def\BibTeX{{\rm B\kern-.05em{\sc i\kern-.025em b}\kern-.08em
    T\kern-.1667em\lower.7ex\hbox{E}\kern-.125emX}}

\begin{document}

\title{Project 2 : Range Doppler Matrix Generation for an LFM Radar System}

\author{\IEEEauthorblockN{Owen Sowatzke}
\IEEEauthorblockA{\textit{Electrical Engineering Department} \\
\textit{University of Arizona}\\
Tucson, USA \\
osowatzke@arizona.edu}}
\maketitle

\begin{abstract}
The range resolution of a radar system is inversely proportional to the bandwidth of its transmitted waveform. To achieve a fine range resolution, a high bandwidth waveform is required. This typically results in an increased ADC sampling rate, which leads to higher production costs. A linear frequency modulated (LFM) waveform can be used in conjunction with stretch processing to provide a large transmitted waveform bandwidth with a reduced ADC sampling rate. This leads to a fine range resolution at reduced production costs. This document provides an overview of an LFM radar system with a focus on the signal processing. It specifically examines the generation of range doppler matrices (RDMs) using range and doppler compression. The effects of amplitude compensation, range windowing, and doppler windowing are also be examined. Finally, the measurement limitations of the LFM radar system are examined and methods for improving these limitations are discussed. 
\end{abstract}

\begin{IEEEkeywords}
Linear Frequency Modulated (LFM) Radar System, Stretch Processing, Range Doppler Matrix, Range Compression, Doppler Compression, Amplitude Compensation, Range Windowing, Doppler Windowing, Range Resolution, Velocity Resolution, Range Ambiguity, Doppler Ambiguity
%Pulse Compression, Linear Frequency Modulated (LFM) Waveform, Matched Filter, SNR, Range Resolution
\end{IEEEkeywords}
\section{Introduction}
In this document, an LFM radar system that leverages stretch processing will be examined. The radar system of interest with configured with the following parameters:
\begin{table}[H]
\caption{Radar System Parameters}
\label{Parameter Table}
\begin{tabularx}{0.5\textwidth}{| X | X |}
\hline
Carrier Frequency & 77 GHz \\
\hline
Sweep Frequency & 76.8 GHz - 77.2 GHz \\
\hline
Pulse Width & 20 $\mu s$ \\
\hline
PRI & 30 $\mu s$ \\
\hline
Fast Time Samples & 400 \\
\hline 
Number of Chirps & 500 \\
\hline
Fast Time Sampling Frequency & 20 MHz \\
\hline
\end{tabularx}
\end{table}
\noindent
The system's ADC data will be generated for a variety of scenarios using MATLAB. Then, signal processing will be performed on this data to generate Range Doppler Matrices (RDMs).
\par
This document will provide a walk-through of the data generation and signal processing with simulation results that highlight key milestones. These milestones include ADC data generation, range and doppler compression, amplitude compensation, and range and doppler windowing. Finally, the measurement limitations of the system will be examined, and methods for improving these limitations will be discussed.
%The LFM radar system will be used to generate Range Doppler Matrices (RDMs). The ADC data will be generated in MATLAB and formatted as a data cube. 
\section{ADC Data Generation}
The target returns for a radar system are delayed and doppler shifted versions of the transmitted signal. For an LFM radar system, the transmitted signal takes the following form:
\begin{equation}
x(t) = e^{j\pi\beta t^2/\tau}
\end{equation}
The instantaneous frequency of this signal is given by the following equation:
\begin{equation}
F_i(t) = \frac{1}{2\pi}\frac{d\theta(t)}{dt} = \frac{\beta}{\tau}t
\end{equation}
If the instantaneous frequency of the transmitted waveform is plotted versus time, the result will be a ramp function. For a target at a constant range (static target), the received signal will be a delayed version of the initial ramp function. If target velocity is added, a DC offset will be added to the delayed ramp function. This is illustrated in Fig. \ref{inst_freq}.
\begin{figure}[H]
\centerline{\fbox{\includegraphics[width=0.4\textwidth]{inst_freq.png}}}
\caption{Instantaneous Frequency of Transmit and Receive Signals.}
\label{inst_freq}
\end{figure}
If the targets returns are sampled directly, a sample rate of at least $\beta$ is required. In the stretch processing receiver, additional analog processing is performed prior to sampling to reduce the sample rate and simplify the signal processing.

In a conventional receiver, the sampling rate would need 
Instead of directly sampling this signal, the stretch processing receiver does additional analog processing. 

The instantaneous frequency of this waveform is given by the following equation:
\begin{equation}
F_i(t) = \frac{1}{2\pi}\frac{d\theta(t)}{dt} = \frac{\beta}{\tau}t
\end{equation}
An LFM radar system transmits the following waveform:
\begin{equation}
x(t) = e^{j\pi\beta t^2/\tau} \quad 0\leq t \leq \tau 
\end{equation}
The bandwidth of this waveform is $\beta$, and the corresponding range resolution is given by:
\begin{equation}
\Delta R = \frac{c}{2\beta}
\end{equation}
A finer range resolution can be achieved by increasing the bandwidth of the LFM waveform. The instantaneous frequency of the LFM waveform is given by the following equation:
\begin{equation}
F_i(t) = \frac{1}{2\pi}\frac{d\theta(t)}{dt} = \frac{\beta}{\tau}t
\end{equation}
As such, the transmitted waveform is a ramp on the time-frequency axis. Similarly, the return for a static target is a delayed version of the same ramp. This is shown in Fig. \ref{tx_rx_plot}.
\begin{figure}[H]%[hbt!]
\centerline{\fbox{\includegraphics[width=0.4\textwidth]{tx_rx_plot.png}}}
\caption{Instantaneous Frequency of Transmit and Receive Signals.}
\label{tx_rx_plot}
\end{figure}
Stretch processing takes advantage of the relationship between the transmit and receive signals. It mixes the receive signal with a signal of the following form:
\begin{equation}
m(t) = e^{-j\pi\beta t^2/\tau}
\end{equation}
This generates a tone at the following beat frequency:
\begin{equation}
F_b = -\frac{\beta t_0}{\tau} = -\frac{2\beta R}{c\tau}
\end{equation}
\begin{figure}[H]%[hbt!]
\centerline{\fbox{\includegraphics[width=0.4\textwidth]{beat_freq.png}}}
\caption{Instantaneous Frequency of Demodulated Tone.}
\label{beat_freq}
\end{figure}
\noindent
A time window $T_w$ can be defined to provide unambiguous coverage of ranges between $0$ and $R_w$. Then, the largest beat frequency magnitude should be
\begin{equation}
|F_w| = \frac{\beta T_w}{\tau} = \frac{2\beta R_w}{c\tau}
\end{equation}
If the time window, $T_w$, is smaller than the pulse length $\tau$, the ADC rate can be smaller than the chirp bandwidth, $\beta$, leading to significant cost savings.
\par
Vectors of ADC samples corresponding to a PRI can be stacked along the slow-time axis to form a data cube. Note that the pulse return will be in the first $L$ samples along the fast-time axis. All additional fast-time samples are noise and can be discarded. The resulting data cube is shown in Fig. \ref{Data_Cube}.
\begin{figure}[H]%[hbt!]
\centerline{\fbox{\includegraphics[width=0.4\textwidth]{Data_Cube.png}}}
\caption{Data Cube.}
\label{Data_Cube}
\end{figure}
\section{Range and Doppler Compression}
An RDM can be generated from the received data cube using range and doppler compression. This is done by taking an FFT along the fast-time axis and another FFT along the slow-time axis.
\begin{figure}[H]
\centerline{\fbox{\includegraphics[width=0.4\textwidth]{RDM.png}}}
\caption{Range and Doppler Compression.}
\label{RDM}
\end{figure}
\noindent
The Fast-Time FFT extracts the frequency content in a single pulse. Each target will produce a single tone, which results in a single peak. The frequency of each peak is influenced by the range and doppler shift of each target. The doppler shift will produce the following error in range measurements:
\begin{equation}
\delta R = -\frac{c\tau F_d}{2\beta}
\end{equation}
The Slow-Time FFT measures doppler shift or velocity of the target. The Slow-Time FFT is not be influenced by the range of the target.
\par
Consider the return from two targets without thermal noise. Let the first target be at a range of $10m$ with a velocity of $15m/s$ and an RCS of $1m^2$. Next, let the second target be at a range of $100m$ with a velocity of $10m/s$ and an RCS of $10000m^2$. Then, the RDM generated by the LFM radar system will be given in Fig. \ref{RDM0}.
%The RDM for the given LFM radar system can be generated for a pair of targets without thermal noise. The received data cube can be generated for a set of targets with ranges of $10m$ and $100m$, velocities of $15m/s$ and $10m/s$, and radar cross sections of $1m^2$ and $10000m^2$. 
\begin{figure}[H]
\centerline{\fbox{\includegraphics[width=0.4\textwidth]{RDM_image0.png}}}
\caption{RDM for Given Scenario.}
\label{RDM0}
\end{figure}
The measured ranges and velocities are approximately equivalent to the ranges and velocities of the targets. Note that the measured range will be influenced by the doppler shift and measurement resolution of the Fast-Time FFT. The doppler (or velocity) axis will be influenced by the measurement resolution of the Slow-Time FFT.
\par
The received power from each target return is related to the target range and RCS by the following equation:
\begin{equation}
P_r \propto \frac{\sigma}{R^4}
\end{equation}
Using the range and RCS of each target, the predicted return power is the same. However, the return of the second target is slightly larger than the first in the RDM. This occurs because the first target return has a greater straddle loss.
%Note that the measured ranges will be shifted by the doppler shift. The measured range of the first target will be shifted by:
%\begin{equation}
%\delta R = -\frac{c\tau F_d}{2\beta}=-\frac{c\tau (2v/\lambda)}{2\beta}=-0.05775m
%\end{equation}
%Similarly, the measured range of the second target will be shifted by:
%\begin{equation}
%\delta R = -\frac{c\tau F_d}{2\beta}=-\frac{c\tau (2v/\lambda)}{2\beta}=-0.0385m
%\end{equation}
%The range axis will be quantized in intervals of 
\section{Amplitude Compensation}
The targets in Fig. \ref{RDM0} may be falsely interpreted as having the same RCS. Amplitude compensation ensures that the return from each target is influenced only by the RCS. Amplitude compensation is performed by scaling each sample output by the fast time FFT by $R^2$, where $R$ is the range that corresponds to a given FFT sample. 
\begin{figure}[H]
\centerline{\fbox{\includegraphics[width=0.4\textwidth]{Amplitude_Compensation.png}}}
\caption{RDM Generation with Amplitude Compensation.}
\end{figure}
Consider two targets returns that are received in the presence of thermal noise with power $0.0001$. Let the range of both targets be $10m$. Let the velocity of the first target be $15m/s$ and the velocity of the second target be $14m/s$. Finally, let the RCS of the first target be $100m^2$ and the RCS of the second target be $1m^2$. Then, the RDM with amplitude compensation is shown in Fig. \ref{RDM1}.
\begin{figure}[H]
\centerline{\fbox{\includegraphics[width=0.4\textwidth]{RDM1.png}}}
\caption{RDM for Given Scenario.}
\label{RDM1}
\end{figure}
Amplitude compensation should ensure that the return of each target is a function of only RCS. Because the RCS of the first target is 20dB greater than that of the second, we expect the first target return to be 20dB greater than the second. This is approximately the case in the RDM. Any discrepancies can be linked to straddle loss.
\section{Range and Doppler Windowing}
In the RDM shown in Fig. \ref{RDM1}, the second target return can be difficult to distinguish from the sidelobes of the first target return. This is even more apparent in the closeup shown in Fig. \ref{RDM1_close}.
\begin{figure}[H]
\centerline{\fbox{\includegraphics[width=0.4\textwidth]{RDM1_close.png}}}
\caption{Closeup of Target Returns.}
\label{RDM1_close}
\end{figure}
\noindent
Windowing the input to the fast-time and slow-time FFT can reduce the sidelobe levels and make the two targets more distinguishable. The signal processing chain with the added FFT windows is shown in Fig. \ref{Full_Chain}.
\begin{figure}[H]
\centerline{\fbox{\includegraphics[width=0.4\textwidth]{Full_Chain.png}}}
\caption{Full Signal Processing Chain with Windowing.}
\label{Full_Chain}
\end{figure}
\noindent
If the targets used to generate Fig. \ref{RDM1} are passed through a radar system with hamming windowing, the resulting RDM is given in Fig. \ref{RDM2}.
\begin{figure}[H]
\centerline{\fbox{\includegraphics[width=0.4\textwidth]{RDM2.png}}}
\caption{RDM for Given Scenario.}
\label{RDM2}
\end{figure}
\noindent
The range and doppler windows increase the mainlobe width and decrease the magnitude of the range and doppler FFTs. This leads to reduced RDM peaks. The sidelobes in Fig. \ref{RDM2} are significantly lower than the sidelobes of Fig. \ref{RDM1}. The reduced sidelobes are even more apparent after zooming on the target returns.
\begin{figure}[H]
\centerline{\fbox{\includegraphics[width=0.4\textwidth]{RDM2_Close.png}}}
\caption{Closeup of Target Returns.}
\label{RDM2}
\end{figure}
\section{Measurement Limitations}
The range and velocity measurements from the RDM are limited by the range resolution, doppler resolution, range ambiguity, and doppler ambiguity of the system. The range resolution of the system is given by the following equation:
\begin{equation}
\Delta R = \frac{c}{2\beta} = 0.375m
\end{equation}
The doppler resolution of the system is given by the following equation:
\begin{equation}
\Delta F_d = \frac{PRF}{M} = \frac{1}{M\cdot PRI} \approx 66.67Hz
\end{equation}
This corresponds to the following velocity resolution :
\begin{equation}
\Delta v = \frac{\Delta F_d \cdot \lambda}{2} = \frac{\Delta F_d \cdot c}{2f_c} \approx 0.1299 m/s
\end{equation}
The range ambiguity of a conventional radar system is the radar corresponding to a PRI. 
\begin{equation}
R_{ua} = \frac{c \cdot PRI}{2} 
\end{equation}
The LFM radar system sacrifices some this unambiguous range for a lower sampling rate. The unambiguous range of an LFM radar system is given by the following equation:
\begin{equation}
R_w = \frac{c T_w}{2} = \frac{c F_s \tau}{2\beta} = 150m
\end{equation}
The unambiguous doppler shift of the system is given by the following equation:
\begin{equation}
F_{ua} = PRF = \frac{1}{PRI} = 50kHz
\end{equation}
The unambiguous velocity is defined as follows:
\begin{equation}
v_{ua} = \frac{PRF\cdot\lambda}{2} = \frac{lambda}{2\cdot PRI} \approx 64.9351m/s 
\end{equation}
If a target is at a range of $200m$ and a velocity of $80m/s$, the target will alias in both velocity in range. If the unambiguous velocity interval is defined as $[0, v_{ua})$, the apparent velocity will be
\begin{equation}
v_a = v - nv_{ua} = 80 - 1 \cdot 64.9351 \approx 15.0649m/s
\end{equation}
The apparent range will be
\begin{equation}
R_a = R - nR_{ua} = 200 - 150 = 50m
\end{equation}
This is shown in Fig. \ref{RDM3}.
\begin{figure}[H]
\centerline{\fbox{\includegraphics[width=0.4\textwidth]{RDM3.png}}}
\caption{RDM for Given Scenario.}
\label{RDM3}
\end{figure}
Note that the measured range is not exactly $50m$ due to range-doppler coupling and the range resolution of the system.
\par
There are a few ways to to increase the unambiguous ranges and velocities of the system. First, the unambiguous range can be increased by increasing the sample rate of the system. For example, to increase the unambiguous range to $300m$, the sample rate should be increased to 
\begin{equation}
F_s = \frac{2\beta R_w}{c\tau} = 40MHz
\end{equation}
Next, the unambiguous velocity can be increased by increasing the $PRF$ of the system. Note that the $PRF$ cannot be increased beyond $1/\tau$. This is because a new pulse cannot be started, while the first pulse is still transmitting. If the PRF is increased to this limit, the resulting unambiguous velocity will be
\begin{equation}
v_{ua} = \frac{PRF\cdot\lambda}{2} = \frac{(1/\tau)\cdot\lambda}{2} = 97.4026 m/s
\end{equation}
Using the adjusted PRF and sample rates, the target at a range of $200m$ and a velocity of $80m/s$ is resolvable without ambiguity. This is shown in Fig. \ref{RDM3_resolvable}.
\begin{figure}[H]
\centerline{\fbox{\includegraphics[width=0.4\textwidth]{RDM3_resolvable.png}}}
\caption{RDM with Target Placed in Unambiguous Location.}
\label{RDM3_resolvable}
\end{figure}
\section{Conclusion}
%A linear frequency modulated (LFM) waveform is a pulse compression waveform defined by the following equation:
%\begin{equation}
%x(t)=e^{j\pi\beta t^2/\tau}
%\label{lfm_waveform}
%\end{equation}
%where:
%\begin{fleqn}[\parindent]
%\begin{align*}
%\beta &= \text{Waveform Bandwidth (Hz)}\\
%\tau &= \text{Waveform Duration (s)}
%\end{align*}
%\end{fleqn}
%A sample pulsed radar system will be created in MATLAB to simulate the matched filter response of this waveform. Using the radar system model, the SNR of the matched filter output will be measured and compared to the SNR of the input signal. Additionally, multiple target returns will be generated and used to measure the system's range resolution.
%\section{Sample Radar System Parameters}
%To generate the required matched filter response, a pulsed radar system was simulated in MATLAB using the following set of parameters: 
%%The pulse radar system simulated in MATLAB was designed with the following set of parameters: 
%\begin{table}[H]
%\caption{Radar System Parameters}
%\label{Parameter Table}
%\begin{tabularx}{0.5\textwidth}{| X | X |}
%\hline
%Carrier Frequency & 10 GHz \\
%\hline
%Sample Rate & 100 MHz \\
%\hline
%Transmit Power & 20 dB \\
%\hline
%Antenna Gain & 44.15 dB \\
%\hline
%Noise Figure & 10 dB \\
%\hline 
%System Losses & 5 dB \\
%\hline
%PRF & 50 kHz \\
%\hline
%Duty Cycle & 20\% \\
%\hline
%Chirp Bandwidth & 100 MHz \\
%\hline
%\end{tabularx}
%\end{table}
%\noindent
%Additional parameters can be computed using the parameters given in Table \ref{Parameter Table}. For example, the radar system's PRI can be computed as follows:
%\begin{equation}
%PRI = \frac{1}{PRF} = 20 {\mu}s
%\label{PRI Equation}
%\end{equation}
%Another parameter of interest, the length of the LFM pulse ($\tau$), can be computed using the duty cycle ($D$) as follows:
%\begin{equation}
%\tau = PRI \cdot D = 4 {\mu}s
%\label{tau equation}
%\end{equation}
%\section{Modeling the Matched Filter Output}
%\label{modeling_section}
%To model the matched filter response of the system, the received signal must first be generated for each of the target returns. The process of generating the received signal is illustrated in Fig. \ref{gen_rx_sig}.
%\begin{figure}[H]
%\centerline{\fbox{\includegraphics[width=0.3\textwidth]{gen_rx_sig.png}}}
%\caption{Generating the Received Signal.}
%\label{gen_rx_sig}
%\end{figure}
%Each of the target returns is a delayed and scaled version of the transmitted signal. As such, the transmitted LFM waveform must be generated as part of the workflow. This is done using equation \eqref{lfm_waveform}. Note that the transmitted signal must be padded with zeros to account for the listening period after transmission.
%\par
%For a target at a range of $R$ meters, the transmitted signal is delayed by
%\begin{equation}
%t_d = \frac{R}{2c} \enspace s
%\end{equation}
%The SNR of the target return is given by the following equation:
%\begin{equation}
%SNR = \frac{P_t G^2 \lambda^2 \sigma}{(4\pi)^3 R^4 k T_0 B_n F_n L_s L_\alpha(R)}
%\label{snr_formula}
%\end{equation}
%To create a target return with the desired SNR, either the signal power or noise power can be scaled. For this simulation, the signal power is scaled while the noise power is held constant. The transmitted signal given in equation \eqref{lfm_waveform} has unit amplitude, and the noise is sampled from the standard complex normal distribution. Therefore, to achieve the desired SNR, the target return must be scaled by the following factor:
%\begin{equation}
%k = 10^{SNR(dB)/20}
%\end{equation}
%\par
%The received signal is the sum of each target return and noise. During waveform transmission, the receiver is assumed to be "blanked" (i.e. it cannot listen while transmitting). To model the effects of "blanking", the beginning of the received pulse is replaced with zeros.
%\par
%To increase the SNR of the received signal, the signal is passed through a matched filter. The radar system's matched filtering process is shown in Fig. \ref{gen_mf_output}. 
%\begin{figure}[H]
%\centerline{\fbox{\includegraphics[width=0.15\textwidth]{gen_mf_output.png}}}
%\caption{Matched Filtering of the Input Signal.}
%\label{gen_mf_output}
%\end{figure}
%\par
%The system's matched filter is generated using to the following equation:
%\begin{equation}
%h = x^*(T_M-t)
%\end{equation}
%To ensure the matched filter is causal, $T_M$ is chosen to be the length of the transmitted LFM waveform.
%\par
%In the radar system model, both the noise and target returns are filtered independently. Note that this provides the same result as directly filtering the received signal.
%\begin{equation}
%y(t) = h(t)*(x(t)+n(t)) = h(t)*x(t) + h(t)*n(t)
%\end{equation}
%However, it also provides the ability to easily extract signal and noise power measurements from the matched filter output.
%\section{Matched Filter Output SNR}
%\label{mf_snr_section}
%The SNR at the output of the matched filter output should be greater than the SNR of the received signal by a factor of $L$, where $L$ is the matched filter length in samples. For the given radar system, the matched filter length is
%\begin{equation}
%L = \frac{\tau}{T_s} = \tau f_s = 400 \text{ samples}
%\end{equation}
%Therefore, the SNR of the matched filter output should be $10log_{10}(400)=26.02 \text{ dB}$ greater than that of the received signal. This result will be confirmed via simulation.
%\par
%To simulate the matched filter response of the system, a target is placed in the first ambiguity at a range corresponding to a time delay of 25\% of the PRI. The resulting range is
%\begin{equation}
%R = \frac{ct_d}{2} = 750 \text{ m}
%\label{tgt1 range}
%\end{equation}
%The RCS of the target can be calculated to provide an arbitrary SNR at the ADC. Starting with equation \eqref{snr_formula}, the RCS can be expressed in the following form:
%\begin{equation}
%\sigma = \frac{(4\pi)^3 R^4 k T_0 B_n F_n L_s L_\alpha(R)\cdot SNR}{P_t G^2 \lambda^2}
%\label{rcs_formula}
%\end{equation}
%For this system, the SNR at the ADC is chosen to be 20 dB. This SNR corresponds to a RCS of $0.01307m^2$.
%\par
%The received signal for this target configuration can be generated according to Section \ref{modeling_section}. The resulting ADC signal is shown in Fig. \ref{adc_sig}.
%\begin{figure}[H]
%\centerline{\fbox{\includegraphics[width=0.4\textwidth]{adc_sig.png}}}
%\caption{Received ADC Signal.}
%\label{adc_sig}
%\end{figure}
%\noindent
%This ADC signal can then be passed through the system's matched filter to produce the plot shown in Fig. \ref{mf_output}.
%\begin{figure}[H]
%\centerline{\fbox{\includegraphics[width=0.4\textwidth]{mf_output.png}}}
%\caption{Matched Filter Output.}
%\label{mf_output}
%\end{figure}
%\noindent
%The SNR of the matched filter output is given by the following equation:
%\begin{equation}
%SNR = 10log_{10}\frac{P_s}{P_n}
%\end{equation}
%where:
%\begin{fleqn}[\parindent]
%\begin{align*}
%P_s &= \text{Signal Power}\\
%P_n &= \text{Noise Power}
%\end{align*}
%\end{fleqn}
%The resulting signal power is defined using the peak of the filtered target return as follows:
%\begin{equation}
%P_s = \text{max(}|x[k]|\text{)}^2
%\end{equation}
%The resulting noise power is defined using the filtered noise samples ($n[k]$). Note that the first $2L-1$ noise samples are ignored due to "blanking" and "charge-up" of the matched filter. Let $m \in [0, N)$ define the "charged-up" noise samples of the matched filter. Then, the noise power is given by
%\begin{equation}
%P_n = \frac{1}{N}\sum_{m=0}^{N-1}|n[m]|^2
%\end{equation}
%\par
%For the given simulation, the observed SNR at the matched filter output is $45.97 \text{ dB}$. Note that this is approximately equivalent to the expected SNR of $46.02 \text{ dB}$.
%\section{Matched Filter Range Resolution} 
%The range resolution of the matched filter output is theoretically given by
%\begin{equation}
%\Delta R = \frac{c}{2\beta} = 1.5 \text{ m}
%\end{equation}
%This result will be confirmed via simulation. First, a target will be placed as described in Section \ref{mf_snr_section}. Then, a second target will be placed at a larger range and moved toward the first target until the two targets are indistinguishable from each other. 
%\par
%To start, the second target was placed two code lengths ($2L$ samples) away from the first target. This guarantees that the matched filter responses from each target return are non-overlapping. The resulting range of the second target is given by
%\begin{equation}
%R = 750 + \frac{c(2\tau)}{2} = 750 + c\tau = 1950 \text{ m}
%\end{equation}
%The SNR of the second target return is configured to be 20 dB. Using equation \eqref{rcs_formula}, the resulting RCS is given by $0.59758m^2$. The matched filter response for the combined target return is shown in Fig. \ref{mf_output_clearly_resolvable}.
%\begin{figure}[H]
%\centerline{\fbox{\includegraphics[width=0.4\textwidth]{clearly_resolvable.png}}}
%\caption{Resolvable Target Returns ($1200$m Spacing).}
%\label{mf_output_clearly_resolvable}
%\end{figure}
%\par
%The second target is clearly distinguishable from the first. As such, the second target can be moved significantly closer to the first target. If it is placed 2 ADC samples away from the first target, the resulting range is given by
%\begin{equation}
%R = 750 + \frac{c(2T_s)}{2} = 750 + cT_s = 753 \text{ m}
%\end{equation}
%The SNR of this target return can be configured as 20 dB. This results in an RCS of $0.01328m^2$. The matched filter response resulting for the updated target returns is shown in Fig. \ref{resolvable_full}.
%\begin{figure}[H]
%\centerline{\fbox{\includegraphics[width=0.4\textwidth]{resolvable_full.png}}}
%\caption{Resolvable Target Returns ($3$m Spacing).}
%\label{resolvable_full}
%\end{figure}
%\noindent
%After zooming in on the peak of the matched filter response, the targets can easily be resolved. This is shown in Fig. \ref{resolvable_peak}.
%\begin{figure}[H]
%\centerline{\fbox{\includegraphics[width=0.4\textwidth]{resolvable_peak.png}}}
%\caption{Resolvable Target Returns ($3$m Spacing).}
%\label{resolvable_peak}
%\end{figure}
%\par
%If the second target is moved one sample closer to the first target, the range of the second target is given by 
%\begin{equation}
%R = 750 + \frac{cT_s}{2} = 751.5 \text{ m}
%\end{equation}
%If the SNR of the target is configured as 20 dB, the resulting RCS is given by $0.01317m^2$. The peak of the updated matched filter response is shown in Fig. \ref{unresolvable_peak}.
%\begin{figure}[H]
%\centerline{\fbox{\includegraphics[width=0.4\textwidth]{unresolvable_peak.png}}}
%\caption{Unresolvable Target Returns ($1.5$m Spacing).}
%\label{unresolvable_peak}
%\end{figure}
%\par
%Note that the target returns have blurred into a single target return. As such, the resulting range resolution is $1.5 \text{ m}$. The measured range resolution is identical to the theoretical range resoution. Therefore, the simulated result confirms the theoretical result.
%\section{Conclusion}
%This document examined the matched filter response of an LFM waveform using a simulated pulsed radar system. Quantities of interest included the SNR and range resolution of the matched filter output. The SNR of the matched filter output was greater than than input SNR by a factor of the code length ($L$). Additionally, the range resolution of the matched filter output was $c/(2\beta)$.
%%A radar system was simulated in MATLAB and used to generate the matched filter response of an LFM waveform. 
%%The measured SNR gain from the matched filter was the code length ($L$), and the measured range resolution was $c/(2\beta)$.
%\par
%For a rectangular pulse of the same length $\tau$, the increase in SNR due to the matched filter is also given by $L$. However, the range resolution is increased to $c\tau/2$. Therefore, the range resolution of an LFM waveform is smaller than that of a rectangular waveform by a factor of $\beta\tau$. For the simulated radar system, $\beta\tau = 400$. This result demonstrates the advantage of using a pulse compression waveform such as LFM. Specifically, the SNR of the system can be increased, while maintaining a fine range resolution.
%
%\onecolumn
%\pagebreak
%\appendices
%\section{MATLAB Source Code}
%\label{matlab_code}
%\lstset{style=Matlab-editor}
%\lstinputlisting{Project1_Sowatzke.m}
%\raggedbottom
%\pagebreak
%\lstinputlisting{radar.m}
%\raggedbottom
\end{document}
